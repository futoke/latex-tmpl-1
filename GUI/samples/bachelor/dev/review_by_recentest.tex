\documentclass[10pt]{article}
\usepackage[utf8]{inputenc}
\usepackage[russian]{babel}
\usepackage[left=2cm,right=2cm,top=2cm,bottom=2cm,bindingoffset=0cm]{geometry}
\usepackage{amsmath}
\usepackage{multirow}
%\usepackage{longtable}
\usepackage{graphicx}

\begin{document}

\begin{center}\textbf{Министерство образования и науки Российской Федерации}\end{center}

\scriptsizeФЕДЕРАЛЬНОЕ ГОСУДАРСТВЕННОЕ АВТОНОМНОЕ ОБРАЗОВАТЕЛЬНОЕ УЧРЕЖДЕНИЕ ВЫСШЕГО ОБРАЗОВАНИЯ

\begin{center}\large\textbf{<<САНКТ-ПЕТЕРБУРГСКИЙ НАЦИОНАЛЬНЫЙ ИССЛЕДОВАТЕЛЬСКИЙ УНИВЕРСИТЕТ ИНФОРМАЦИОННЫХ ТЕХНОЛОГИЙ, МЕХАНИКИ И ОПТИКИ>>}\end{center}

\begin{center}
\textbf{\largeОТЗЫВ РЕЦЕНЗЕНТА
~\\О ВЫПУСКНОЙ КВАЛИФИКАЦИОННОЙ РАБОТЕ}
\end{center}

\large
~\\\textbf{Студент }$\underset{\text{(ФИО)}}{\underline{\hspace{0\textwidth\text{[STUDENT]}}}}$\quad$\underset{\text{(ПОДПИСЬ)}}{\underline{\hspace{0\textwidth\text{\hspace{7em}}}}}$
\textbf{Группа }\underline{[GROUP]} \textbf{Кафедра }\underline{[DEPARTMENT]} \textbf{Факультет} \underline{[FACULTY]}
~\\\textbf{Квалификация} $\underset{\text{(бакалавр, магистр, специалист)}}{\underline{\hspace{0\textwidth\text{БАКАЛАВР}}}}$
~\\~\\\textbf{Наименование темы: }\underline{[SUBJECT]}
~\\~\\\textbf{Руководитель }\underline{[LEADER]}
\begin{center}\textbf{ОЦЕНКА ВЫПУСКНОЙ КВАЛИФИКАЦИОННОЙ РАБОТЫ}
\end{center}

\normalsize
\begin{table}[!h]
\begin{center}
\begin{tabular}{|c|c|l|c|c|c|c|c|}
\hline
\multirow{1}{*}{}& № & \textbf{Показатели оценки} & \multicolumn{5}{|c|}{\textbf{Оценка}}  \\ \cline{4-8}
& & & 5& 4& 3& 2& 0* \\ \hline
\multirow{9}{*}{\rotatebox{90}{Справочно-информационная\quad}}& 1 & Соответствие представленного материала техническому заданию & & & & & \\ \cline{2-8}
& 2 & Раскрытие актуальности тематики работы & & & & & \\ \cline{2-8}
& 3 & Степень полноты обзора состояния вопроса  & & & & & \\ \cline{2-8}
& 4 & Корректность постановки задачи исследования и разработки & & & & & \\ \cline{2-8}
& 5 & \multirow{1}{*}{Уровень и корректность использования в работе методов исследований,} & & & & & \\
& & математического моделирования, инженерных расчетов& & & & & \\ \cline{2-8}
& 6 & \multirow{1}{*}{Степень комплексности работы, применение в ней знаний} & & & & & \\
& & естественнонаучных, социально-гуманитарных, экономических, обще  & & & & & \\ 
& & профессиональных и специальных дисциплин & & & & & \\ \cline{2-8}
& 7 & Использование информационных ресурсов Internet & & & & & \\ \cline{2-8}
& 8 & Использование современных пакетов компьютерных программ и технологий	& & & & & \\ \cline{2-8}
& 9 & \multirow{1}{*}{Наличие публикаций, участие в н.-т. конференциях, награды за участие в} & & & & & \\ 
& & конкурсах, подтвержденных копиями & & & & & \\ \hline
\multirow{2}{*}{\rotatebox{90}{Творческая\hspace{0.5em}}}& 10 & \multirow{1}{*}{Оригинальность и новизна полученных результатов, научных} & & & & & \\		
& & конструкторских и технологических решений	 & & & & & \\
& & & & & & & \\ \cline{2-8}
& 11 & Ясность, четкость, последовательность и обоснованность изложения	& & & & & \\ 
& & & & & & & \\ \hline
\multirow{8}{*}{\rotatebox{90}{Оформительская}}& \multirow{4}{*}{}12 & Уровень оформления пояснительной записки: & & & & & \\ \cline{3-8}
& &\hspace{1.5em} -	общий уровень грамотности & & & & & \\ \cline{3-8}
& &\hspace{1.5em} -	стиль изложения& & & & & \\ \cline{3-8}
& &\hspace{1.5em} -	 качество иллюстраций	& & & & & \\ \cline{2-8}
& 13 & \multirow{1}{*}{Объем и качество выполнения графического материала, его соответствие} & & & & & \\
& & тексту записки & & & & & \\ \cline{2-8}
& 14 & \multirow{1}{*}{Соответствие требованиям стандарта оформления пояснительной записки и} & & & & & \\
& & графического материала& & & & & \\ \hline
\multicolumn{3}{|c|}{\textbf{ИТОГОВАЯ ОЦЕНКА}} & \multicolumn{5}{|c|}{} \\ \hline
\end{tabular}
\end{center}
\hspace{4em}* - не оценивается (трудно оценить)
\end{table}

\large
~\\~\\\textbf{Отмеченные достоинства } [ADVANTAGES]
~\\~\\\textbf{Отмеченные недостатки } [DISADVANTAGES]
~\\~\\\textbf{Заключение: } Считаю, что ВКР студента \underline{[STUDENT]} на тему:
~\\~\\$\underset{\text{(название выпускной квалификационной работы)}}{\underline{\hspace{0\textwidth\text{<<[TOPIC]>>}}}}$
соответствует требованиям Университета ИТМО, предъявляемым к ВКР и заслуживает оценки \underline{[MARK]}, а её автор присуждения квалификации
$\underset{\text{(бакалавр, инженер, магистр)}}{\underline{\hspace{0\textwidth\text{\qquadБАКАЛАВР\qquad}}}}$ по направлению подготовки \underline{[DIRECTION]}

~\\~\\Рецензент $\underset{\text{(ПОДПИСЬ)}}{\underline{\hspace{0\textwidth\text{\qquad\qquad\qquad}}}}$ \qquad\qquad$\underset{\text{(ФИО)}}{\underline{\hspace{0\textwidth\text{[RECENTEST]}}}}$

~\\<<\underline{[DAY-REVIEW-RECENTEST]}>>\underline{[MONTH-REVIEW-RECENTEST] [YEAR-REVIEW-RECENTEST] г.}

\end{document}