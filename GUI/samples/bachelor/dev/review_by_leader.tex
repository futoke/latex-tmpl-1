\documentclass[10pt]{article}
\usepackage[utf8]{inputenc}
\usepackage[russian]{babel}
\usepackage[left=2cm,right=2cm,top=2cm,bottom=2cm,bindingoffset=0cm]{geometry}
\usepackage{amsmath}
\usepackage{multirow}
\usepackage{graphicx}


\begin{document}

\begin{center}\textbf{Министерство образования и науки Российской Федерации}\end{center}

\scriptsizeФЕДЕРАЛЬНОЕ ГОСУДАРСТВЕННОЕ АВТОНОМНОЕ ОБРАЗОВАТЕЛЬНОЕ УЧРЕЖДЕНИЕ ВЫСШЕГО ОБРАЗОВАНИЯ

\begin{center}\large\textbf{<<САНКТ-ПЕТЕРБУРГСКИЙ НАЦИОНАЛЬНЫЙ ИССЛЕДОВАТЕЛЬСКИЙ УНИВЕРСИТЕТ ИНФОРМАЦИОННЫХ ТЕХНОЛОГИЙ, МЕХАНИКИ И ОПТИКИ>>}\end{center}

\begin{center}
\textbf{\largeОТЗЫВ РУКОВОДИТЕЛЯ
~\\О ВЫПУСКНОЙ КВАЛИФИКАЦИОННОЙ РАБОТЕ}
\end{center}

\large
~\\\textbf{Студент }$\underset{\text{(ФИО)}}{\underline{\hspace{0\textwidth\text{[STUDENT]}}}}$ \quad$\underset{\text{(ПОДПИСЬ)}}{\underline{\hspace{0\textwidth\text{\hspace{5em}}}}}$

\textbf{Группа }\underline{[GROUP]} \textbf{Кафедра }\underline{[DEPARTMENT]} \textbf{Факультет} \underline{[FACULTY]}
~\\\textbf{Квалификация} $\underset{\text{(бакалавр, магистр, специалист)}}{\underline{\hspace{0\textwidth\text{БАКАЛАВР}}}}$
~\\~\\\textbf{Наименование темы: }\underline{[TOPIC]}
~\\~\\\textbf{Руководитель }$\underset{\text{(Фамилия, И., О., место работы, должность, ученое звание, степень)}}{\underline{\hspace{0\textwidth\text{[LEADER]}}}}$
\begin{center}\textbf{ПОКАЗАТЕЛИ ОЦЕНКИ ВЫПУСКНОЙ КВАЛИФИКАЦИОННОЙ РАБОТЫ}
\end{center}
\normalsize
\begin{table}[!h]
\begin{center}
\begin{tabular}{|c c|c|l|c|c|c|c|}
\hline
& & № & \textbf{Показатели оценки} & \multicolumn{4}{|c|}{\textbf{Оценка}}  \\ \cline{5-8}
& & 1 & & 5& 4& 3& 0* \\ \hline


\multirow{5}{0.7em}{\rotatebox{90}{\smallПрофессиональная}}& & 2 & \multirow{1}{*}{Оригинальность и новизна полученных результатов, научных, } & & & &\\ 
& &  & конструкторских и технологических решений & & & & \\ \cline{3-8}
& & 3 & Степень полноты обзора, обобщения, анализа, систематизации  & & & & \\ \cline{3-8}
& & 4 & Степень самостоятельного и творческого участия студента в работе & & & & \\ \cline{3-8}
& & 5 & Корректность формулирования цели и задачи исследования и разработки  & & & & \\ \cline{3-8}
& & 6 & \multirow{1}{*}{Уровень и корректность использования в работе современных методов } & & & & \\
& & & исследований, математического моделирования, инженерных расчетов & & & & \\ \hline


\multirow{3}{0.7em}{\rotatebox{90}{\smallСправочно-\hspace{2.6em}}}& \multirow{3}{0.7em}{\rotatebox{90}{\smallинформационная}}& 7 & \multirow{1}{*}{Степень комплексности работы. Применение в ней знаний } & & & & \\		
& & & естественнонаучных, социально-гуманитарных и экономических, & & & & \\
& & & общепрофессиональных и специальных дисциплин & & & & \\ \cline{3-8}

& & 8 & \multirow{4}{*}{}Использование современных пакетов компьютерных программ и  & & & & \\ 
& & & технологий & & & & \\ \cline{3-8}
& & 9 & \multirow{1}{*}{Наличие публикаций, участие в н.-т. конференциях, награды за участие в} & & & & \\
& & & конкурсах& & & & \\ \hline

\multirow{3}{*}{\rotatebox{90}{\smallОформительская  }}& & 10 &\multirow{1}{*}{Ясность, четкость, последовательность и обоснованность изложения }& & & & \\
& & & пояснительной записки & & & & \\ \cline{3-8}
& & 11 & \multirow{1}{*}{Качество оформления пояснительной записки (общий уровень } & & & & \\
& & & грамотности, стиль изложения, качество иллюстраций, соответствие  & & & & \\
& & & требованиям стандарта) & & & & \\ \cline{3-8}
& & 12 & \multirow{1}{*}{Объем и качество выполнения графического материала, его соответствие } & & & &\\
& & & тексту записки и стандартам & & & & \\ \hline

\multicolumn{4}{|c|}{\textbf{ИТОГОВАЯ ОЦЕНКА}} & \multicolumn{4}{|c|}{} \\ \hline
\end{tabular}
\end{center}
\hspace{4em}* - не оценивается (трудно оценить)
\end{table}



~\\\textbf{Отмеченные достоинства: }\textit{(например:1 способность к формулированию цели, задачи и плана научного исследования в профессиональной области;
2 способность к построению математических моделей объектов исследования и выбору численного метода их моделирования, разработке нового или выбор готового алгоритма решения задач;
3 способность к выбору оптимального метода и разработке программ экспериментальных исследований, проведению измерений в профессиональной области и т.д. }
[ADVANTAGES]
~\\~\\\textbf{Отмеченные недостатки } [DISADVANTAGES]
~\\~\\\textbf{Заключение: } Считаю, что ВКР студента \underline{[STUDENT]} на тему:
~\\$\underset{\text{(название выпускной квалификационной работы)}}{\underline{\hspace{0\textwidth\text{<<[TOPIC]>>}}}}$
соответствует требованиям Университета ИТМО, предъявляемым к ВКР и заслуживает оценки \underline{[MARK]}, а её автор присуждения квалификации
$\underset{\text{(бакалавр, инженер, магистр)}}{\underline{\hspace{0\textwidth\text{БАКАЛАВР}}}}$ по направлению подготовки \underline{[DIRECTION]}

~\\~\\Руководитель $\underset{\text{(ПОДПИСЬ)}}{\underline{\hspace{0\textwidth\text{\qquad\qquad\qquad}}}}$ \qquad\qquad$\underset{\text{(ФИО)}}{\underline{\hspace{0\textwidth\text{[LEADER]}}}}$

~\\<<\underline{[DAY-REVIEW-RECENTEST]}>>\underline{[MONTH-REVIEW-RECENTEST] [YEAR-REVIEW-RECENTEST] г.}


\end{document}